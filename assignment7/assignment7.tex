\documentclass[12pt]{article}
\usepackage{fontspec}   %加這個就可以設定字體
\usepackage{xeCJK}       %讓中英文字體分開設置
\setmainfont{Times New Roman}
\setCJKmainfont{標楷體} %設定中文為系統上的字型,而英文不去更動,使用原TeX字型
\XeTeXlinebreaklocale "zh"             %這兩行一定要加,中文才能自動換行
\XeTeXlinebreakskip = 0pt plus 1pt     %這兩行一定要加,中文才能自動換行
\usepackage{amsmath, amsthm, amssymb} %引入數學符號的套件,例如實數R、定理Thm...
\usepackage{graphicx}                 %現在, 假設我們要插入 pic.png 這個圖檔, 使用
%\title{我是標題}
%\author{我是作者}
%\date{} %不要日期

\newcommand{\uA}       {\mbox{\boldmath$A$}}
\usepackage{textcomp}
\usepackage{array}
\usepackage{graphicx}
\usepackage{colortbl}
\usepackage{color,xcolor}
\usepackage{listings}
\usepackage{array,booktabs}   %這三個為表格使用的套件
\usepackage{textpos}
\usepackage{float}
\usepackage{listings}

\title{Statistical learning assignment 7- chapter 4}
\author{孫浩哲 \hspace{0.7cm} M072040002}
\date{November 1, 2018}
\begin{document}
\maketitle
\begin{itemize}
\item[4.4]
(a)\\
$10\%$, because we choose\ $10\%$ of the range of\ $X$.\\[2ex]
(b)\\
$10\%\times10\%=1\%$\\[2ex]
(c)\\
$(0.1)^{100}\times100\%=10^{-98}\%$\\[2ex]
(d)\\
As\ $p$ increases, the training observations "near" may decreases in power.\\[2ex]
(e)\\  
$p=1, l=10\%^1=0.1$\\[2ex]
$p=2, l=\sqrt{10\%}\approx0.3162$\\[2ex]
$p=100, l=\sqrt[100]{10\%}\approx1$
\item[4.5]
(a)\\
We expect QDA is better on training set because QDA is more flexible, and we expect LDA is better on test set because we want to avoid overfitting.\\[2ex]
(b)\\
Because it is non-linear, so we prefer QDA on both sets.\\[2ex]
(c)\\
Improve, because as the sample size increases, the model may be more complicated, so we need a more flexible model to make a better fit.\\[2ex]
(d)\\
False, the QDA may be overfitting if the sample size is small.
\item[4.6]
(a)\\
$$
P(X)=\frac{e^{(-6+0.05\times40+3.5)}}{1+e^{(-6+0.05\times40+3.5)}}\approx37.75\%
$$\\[3ex]
(b)\\
\begin{align*}
-6+3.5+0.05\times hours
&=0\rightarrow hours=50\\
&(\because\frac{e^0}{1+e^0}=0.5)
\end{align*}
\end{itemize}
\end{document}