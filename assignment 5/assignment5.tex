\documentclass[12pt]{article}
\usepackage{fontspec}   %加這個就可以設定字體
\usepackage{xeCJK}       %讓中英文字體分開設置
\setmainfont{Times New Roman}
\setCJKmainfont{標楷體} %設定中文為系統上的字型,而英文不去更動,使用原TeX字型
\XeTeXlinebreaklocale "zh"             %這兩行一定要加,中文才能自動換行
\XeTeXlinebreakskip = 0pt plus 1pt     %這兩行一定要加,中文才能自動換行
\usepackage{amsmath, amsthm, amssymb} %引入數學符號的套件,例如實數R、定理Thm...
\usepackage{graphicx}                 %現在, 假設我們要插入 pic.png 這個圖檔, 使用
%\title{我是標題}
%\author{我是作者}
%\date{} %不要日期

\newcommand{\uA}       {\mbox{\boldmath$A$}}
\usepackage{textcomp}
\usepackage{array}
\usepackage{graphicx}
\usepackage{colortbl}
\usepackage{color,xcolor}
\usepackage{listings}
\usepackage{array,booktabs}   %這三個為表格使用的套件
\usepackage{textpos}
\usepackage{float}
\usepackage{listings}

\title{Statistical learning assignment 5- chapter 3}
\author{孫浩哲 \hspace{0.7cm} M072040002}
\date{October 18, 2018}
\begin{document}
\maketitle
\begin{itemize}
\item[2.]\ \\
The concepts of KNN classifier\&KNN regression are similar.\\[2ex]
The main difference between the two methods is that the output of KNN classifier is qualitative, but the output of another one is quantitative. 
\item[7.]\ \\
\begin{align*}
R^2=\frac{SSR}{SST}
&=\frac{\sum_{i=1}^{n}(\hat{y_{i}}-\bar{y})^2}{\sum_{i=1}^{n}(y_{i}-\bar{y})^2}\\
&=\frac{\sum_{i=1}^{n}(b_{0}+b_{1}x_{i}-\bar{y})^2}{\sum_{i=1}^{n}(y_{i}-\bar{y})^2}\\
&=\frac{\sum_{i=1}^{n}(\bar{y}-b_{1}\bar{x}+b_{1}x_{i}-\bar{y})^2}{\sum_{i=1}^{n}(y_{i}-\bar{y})^2}\\
&=\frac{\sum_{i=1}^{n}(b_{1}x_{i}-b_{1}\bar{x})^2}{\sum_{i=1}^{n}(y_{i}-\bar{y})^2}\\
&=b_{1}^2\frac{S_{xx}}{s_{yy}}=(\frac{S_{xy}}{S_{xx}})^2\frac{S_{xx}}{s_{yy}}\\
&=\frac{(S_{xy})^2}{S_{xx}{S_{yy}}}=[cor(x,y)]^2
\end{align*}
\end{itemize}
\end{document}